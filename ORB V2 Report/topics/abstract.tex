\begin{center}
    \Large \textbf{ABSTRACT}\\
\end{center}
\vspace{0.5cm} 

\textit{Keywords: Quadruped, Autonomous, Path Planning, Servo, Computer Vision, Inverse Kinematics, LiDAR}
\vspace{0.5cm}

Quadruped robots are highly efficient and have several advantages when compared to other wheeled and two-legged robots. The fact that it has 4 legs for locomotion creates extra possibilities for movement, stability and dynamic maneuvarability. This proves to be perfect for navigating complex terrain. Additionally, the low center of gravity of these robots provides more stability and balance to its movement. Its design and gait pattern are heavily inspired from that of four-legged animals. Our project implements the movement functionalities of the quadruped and its navigation through an environment. Visual perception is done by means of computer vision coded in python for the purpose of autonomous navigation. A Raspberry Pi is used as the main brain of the robot, and an Arduino Mega controls the actuators. Several navigation routines are solved using path planning algorithms, stair detection algorithms, visual odometry etc. Inverse Kinematics enables endpoint control of each of the four legs, computing the angles for leg joints for the actuation of the servos. Stereo Camera Setups, ultrasonic sensors and a LiDAR unit is used for environment perception and mapping, and the data from these are used for the overall perception of the environment and navigation of the robot.