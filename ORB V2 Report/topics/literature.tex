\chapter{Literature Survey}

Our project to design and develop a quadruped robot was inspired by the Boston Dynamics Spot\textsuperscript{\textregistered} robot.

\begin{enumerate}
    \item Self Driving Car Specialization - \emph{University of Toronto}: This course gave us a comprehensive understanding of state-of-the-art engineering practices used in the Autonomous Industry. Interacted with real data sets from an autonomous vehicle (AV) - all through hands-on projects using simulatorions.Helped us to Understand LIDAR scan matching and the Iterative Closest Point algorithm  Learned various algorithms related to Visual perception, Visual odometry,State Estimation,static and dynamic object detection, Localization and autonomous navigation.
    
     This also Enabled us to to find ways to get the shortest path over a graph or road network using Dijkstra's and the A* algorithm, use finite state machines to select safe behaviors to execute, and design optimal, smooth paths and velocity profiles to navigate safely around obstacles while obeying traffic laws and to also build occupancy grid maps of static elements in the environment and learn how to use them for efficient collision checking.This course gaves an overall idea about the ability to construct a full self-driving planning solution, keeping the vehicle safe at all times. Many of the learnings from this course have been highly usefull for building this project.
 
    \item Robotic Path planning A* and D* - \emph{Carnegie Mellon}: Thwe exploredse studies underlying algorithmic techniques used for planning and decision-making in robotics and examines case studies in ground and aerial robots, humanoids, mobile manipulation platforms and multi-robot systems. Planning and Decision-making are critical components of autonomy in robotic systems. These components are responsible for making decisions that range from path planning and motion planning to coverage and task planning to taking actions that help robots understand the world around them better. This course we explored various underlying algorithmic techniques used for planning and decision-making in robotics and examines case studies in ground and aerial robots, humanoids, mobile manipulation platforms and multi-robot systems.

    \item Boston Dynamics spot Documentation : From this documentation we did Exploratory study and it gave us the  inspiration for building quadruped robots and learn about their various applications.Spot is an agile mobile robot that navigates terrain with unprecedented mobility, allowing you to automate routine inspection tasks and data capture safely, accurately, and frequently. The results are Safer, more efficient and more predictable operations and solutions in the field of robotic applications. It has 360° Perception and dynamic balance.Spot’s base platform provides advanced mobility and perception to navigate stairs, gravel, and rough terrain while collecting 2D and 3D information with on board-sensors. We can also add payloads provided by Boston Dynamics or third-parties to enhance Spot’s sensing and data processing capabilities 

    \item Hello Real World With ROS : Robotic Operating System is a meta operating system which runs on top of other Operating systems like linux. It's beneficial in the field of robotics, because it includes several packages and functons that are helpfull in the development of robots. Its speccially dedicated for robotic systems. this course helped us in understanding the application of ROS for Robot programming and Simulation. We also learned how to use ROS communication tools (topics, services, actions) to exchange information between functional modules, Introduction simulation environment. This enalbed us to run simulations for the robot using Rviz and Gazebo, Simulation features available in ROS packages for visualization and creation of a custom environment with a robot
        Mapping of the robot environment and navigation with a mobile robot.

    \item The simulated control framework is based on "\emph{Hierarchical controller for highly dynamic locomotion utilizing pattern modulation and impedance control : implementation on the MIT Cheetah robot}".This thesis presents a hierarchical control algorithm for quadrupedal locomotion. We address three challenges in developing a controller for high-speed running: locomotion stability, control of ground reaction force, and coordination of four limbs. To tackle these challenges, the proposed algorithm employs three strategies. Leg impedance control provides programmable virtual compliance of each leg which achieve self-stability in locomotion. The four legs exert forces to the ground using equilibrium-point hypothesis. A gait pattern modulator imposes a desired footfall sequence. The control algorithm is verified in a dynamic simulator constructed using MATLAB and then in the subsequent experiments on the MIT Cheetah robot. The experiments on the MIT Cheetah robot demonstrates high speed trot running reaching up to the speed of 6 m/s on a treadmill. This speed corresponds to a Froude number (Fr = 7.34), which is comparatively higher than other existing quadrupedal robots.

\end{enumerate}